% Options for packages loaded elsewhere
\PassOptionsToPackage{unicode}{hyperref}
\PassOptionsToPackage{hyphens}{url}
%
\documentclass[
]{article}
\usepackage{lmodern}
\usepackage{amssymb,amsmath}
\usepackage{ifxetex,ifluatex}
\ifnum 0\ifxetex 1\fi\ifluatex 1\fi=0 % if pdftex
  \usepackage[T1]{fontenc}
  \usepackage[utf8]{inputenc}
  \usepackage{textcomp} % provide euro and other symbols
\else % if luatex or xetex
  \usepackage{unicode-math}
  \defaultfontfeatures{Scale=MatchLowercase}
  \defaultfontfeatures[\rmfamily]{Ligatures=TeX,Scale=1}
\fi
% Use upquote if available, for straight quotes in verbatim environments
\IfFileExists{upquote.sty}{\usepackage{upquote}}{}
\IfFileExists{microtype.sty}{% use microtype if available
  \usepackage[]{microtype}
  \UseMicrotypeSet[protrusion]{basicmath} % disable protrusion for tt fonts
}{}
\makeatletter
\@ifundefined{KOMAClassName}{% if non-KOMA class
  \IfFileExists{parskip.sty}{%
    \usepackage{parskip}
  }{% else
    \setlength{\parindent}{0pt}
    \setlength{\parskip}{6pt plus 2pt minus 1pt}}
}{% if KOMA class
  \KOMAoptions{parskip=half}}
\makeatother
\usepackage{xcolor}
\IfFileExists{xurl.sty}{\usepackage{xurl}}{} % add URL line breaks if available
\IfFileExists{bookmark.sty}{\usepackage{bookmark}}{\usepackage{hyperref}}
\hypersetup{
  pdftitle={Getting Started with RMySQL},
  pdfauthor={Joseph Crispell},
  hidelinks,
  pdfcreator={LaTeX via pandoc}}
\urlstyle{same} % disable monospaced font for URLs
\usepackage[margin=1in]{geometry}
\usepackage{color}
\usepackage{fancyvrb}
\newcommand{\VerbBar}{|}
\newcommand{\VERB}{\Verb[commandchars=\\\{\}]}
\DefineVerbatimEnvironment{Highlighting}{Verbatim}{commandchars=\\\{\}}
% Add ',fontsize=\small' for more characters per line
\usepackage{framed}
\definecolor{shadecolor}{RGB}{248,248,248}
\newenvironment{Shaded}{\begin{snugshade}}{\end{snugshade}}
\newcommand{\AlertTok}[1]{\textcolor[rgb]{0.94,0.16,0.16}{#1}}
\newcommand{\AnnotationTok}[1]{\textcolor[rgb]{0.56,0.35,0.01}{\textbf{\textit{#1}}}}
\newcommand{\AttributeTok}[1]{\textcolor[rgb]{0.77,0.63,0.00}{#1}}
\newcommand{\BaseNTok}[1]{\textcolor[rgb]{0.00,0.00,0.81}{#1}}
\newcommand{\BuiltInTok}[1]{#1}
\newcommand{\CharTok}[1]{\textcolor[rgb]{0.31,0.60,0.02}{#1}}
\newcommand{\CommentTok}[1]{\textcolor[rgb]{0.56,0.35,0.01}{\textit{#1}}}
\newcommand{\CommentVarTok}[1]{\textcolor[rgb]{0.56,0.35,0.01}{\textbf{\textit{#1}}}}
\newcommand{\ConstantTok}[1]{\textcolor[rgb]{0.00,0.00,0.00}{#1}}
\newcommand{\ControlFlowTok}[1]{\textcolor[rgb]{0.13,0.29,0.53}{\textbf{#1}}}
\newcommand{\DataTypeTok}[1]{\textcolor[rgb]{0.13,0.29,0.53}{#1}}
\newcommand{\DecValTok}[1]{\textcolor[rgb]{0.00,0.00,0.81}{#1}}
\newcommand{\DocumentationTok}[1]{\textcolor[rgb]{0.56,0.35,0.01}{\textbf{\textit{#1}}}}
\newcommand{\ErrorTok}[1]{\textcolor[rgb]{0.64,0.00,0.00}{\textbf{#1}}}
\newcommand{\ExtensionTok}[1]{#1}
\newcommand{\FloatTok}[1]{\textcolor[rgb]{0.00,0.00,0.81}{#1}}
\newcommand{\FunctionTok}[1]{\textcolor[rgb]{0.00,0.00,0.00}{#1}}
\newcommand{\ImportTok}[1]{#1}
\newcommand{\InformationTok}[1]{\textcolor[rgb]{0.56,0.35,0.01}{\textbf{\textit{#1}}}}
\newcommand{\KeywordTok}[1]{\textcolor[rgb]{0.13,0.29,0.53}{\textbf{#1}}}
\newcommand{\NormalTok}[1]{#1}
\newcommand{\OperatorTok}[1]{\textcolor[rgb]{0.81,0.36,0.00}{\textbf{#1}}}
\newcommand{\OtherTok}[1]{\textcolor[rgb]{0.56,0.35,0.01}{#1}}
\newcommand{\PreprocessorTok}[1]{\textcolor[rgb]{0.56,0.35,0.01}{\textit{#1}}}
\newcommand{\RegionMarkerTok}[1]{#1}
\newcommand{\SpecialCharTok}[1]{\textcolor[rgb]{0.00,0.00,0.00}{#1}}
\newcommand{\SpecialStringTok}[1]{\textcolor[rgb]{0.31,0.60,0.02}{#1}}
\newcommand{\StringTok}[1]{\textcolor[rgb]{0.31,0.60,0.02}{#1}}
\newcommand{\VariableTok}[1]{\textcolor[rgb]{0.00,0.00,0.00}{#1}}
\newcommand{\VerbatimStringTok}[1]{\textcolor[rgb]{0.31,0.60,0.02}{#1}}
\newcommand{\WarningTok}[1]{\textcolor[rgb]{0.56,0.35,0.01}{\textbf{\textit{#1}}}}
\usepackage{graphicx}
\makeatletter
\def\maxwidth{\ifdim\Gin@nat@width>\linewidth\linewidth\else\Gin@nat@width\fi}
\def\maxheight{\ifdim\Gin@nat@height>\textheight\textheight\else\Gin@nat@height\fi}
\makeatother
% Scale images if necessary, so that they will not overflow the page
% margins by default, and it is still possible to overwrite the defaults
% using explicit options in \includegraphics[width, height, ...]{}
\setkeys{Gin}{width=\maxwidth,height=\maxheight,keepaspectratio}
% Set default figure placement to htbp
\makeatletter
\def\fps@figure{htbp}
\makeatother
\setlength{\emergencystretch}{3em} % prevent overfull lines
\providecommand{\tightlist}{%
  \setlength{\itemsep}{0pt}\setlength{\parskip}{0pt}}
\setcounter{secnumdepth}{-\maxdimen} % remove section numbering

\title{Getting Started with RMySQL}
\author{Joseph Crispell}
\date{17 Apr 2020}

\begin{document}
\maketitle

{
\setcounter{tocdepth}{2}
\tableofcontents
}
\hypertarget{introduction}{%
\section{Introduction}\label{introduction}}

Usually a Reproducible Analytical Pipeline will start by retrieving data
from a database. Often that database could be a \texttt{SQL} database.
\href{https://www.mysql.com/}{\texttt{MySQL}} is an open-source
relational database management system that uses \emph{S}tructured
\emph{Q}uery \emph{L}anguage (SQL). We can use \texttt{MySQL} to store
complex data tables that relate to one another.

\texttt{RMySQL} is an R package that we can use to interact with
\texttt{MySQL}, writing queries, updating data and creating new tables.

The current tutorial aims to give a brief introduction to using
\texttt{RMySQL} in R. The current content is more-or-less the same as
the TutorialsPoint
\href{https://www.tutorialspoint.com/r/r_database.htm}{tutorial}, with a
few updates and edits to match updates to \texttt{RMySQL}.

\hypertarget{installing-mysql}{%
\section{\texorpdfstring{Installing
\texttt{MySQL}}{Installing MySQL}}\label{installing-mysql}}

To get started, you'll ned to install \texttt{MySQL}, here are links for
the different operating systems:

\begin{itemize}
\tightlist
\item
  Installing on linux systems
  (\href{https://dev.mysql.com/doc/refman/8.0/en/linux-installation.html}{link}),
  with extra help
  \href{https://itsfoss.com/install-mysql-ubuntu/}{here}.
\item
  Installing on mac:
  (\href{https://dev.mysql.com/doc/mysql-osx-excerpt/5.7/en/osx-installation.html}{link}),
  also, I think it comes ready installed! See
  \href{https://www.thoughtco.com/installing-mysql-on-mac-2693866}{here})
\item
  Installing on Windows
  (\href{https://dev.mysql.com/downloads/installer/}{link}), with extra
  help
  \href{https://www.wikihow.com/Install-the-MySQL-Database-Server-on-Your-Windows-PC}{here}
\end{itemize}

For this tutorial, I worked on a Windows machine and it took me a wee
while to get MySQL installed properly, so be patient.

\hypertarget{installing-rmysql}{%
\section{\texorpdfstring{Installing
\texttt{RMySQL}}{Installing RMySQL}}\label{installing-rmysql}}

Much easier! Just use the following code:

\begin{Shaded}
\begin{Highlighting}[]
\KeywordTok{install.packages}\NormalTok{(}\StringTok{"RMySQL"}\NormalTok{)}
\end{Highlighting}
\end{Shaded}

Then we can load the library:

\begin{Shaded}
\begin{Highlighting}[]
\KeywordTok{library}\NormalTok{(RMySQL)}
\end{Highlighting}
\end{Shaded}

\begin{verbatim}
## Warning: package 'RMySQL' was built under R version 3.6.3
\end{verbatim}

\begin{verbatim}
## Loading required package: DBI
\end{verbatim}

\hypertarget{working-with-rmysql}{%
\section{\texorpdfstring{Working with
\texttt{RMySQL}}{Working with RMySQL}}\label{working-with-rmysql}}

\hypertarget{create-a-connection-to-mysql}{%
\subsection{Create a connection to
MySQL}\label{create-a-connection-to-mysql}}

We are going to connect to the ``sakila'' database, which comes with the
\texttt{MySQL} installation:

\begin{Shaded}
\begin{Highlighting}[]
\CommentTok{\# Open a connection to a MySQL database}
\NormalTok{connection \textless{}{-}}\StringTok{ }\KeywordTok{dbConnect}\NormalTok{(}\KeywordTok{MySQL}\NormalTok{(), }
                            \DataTypeTok{user=}\StringTok{\textquotesingle{}root\textquotesingle{}}\NormalTok{, }
                            \DataTypeTok{password=}\KeywordTok{readline}\NormalTok{(}\DataTypeTok{prompt=}\StringTok{"Enter password: "}\NormalTok{), }\CommentTok{\# Doing this as password never stored in easily accessible format now}
                            \DataTypeTok{dbname=}\StringTok{\textquotesingle{}sakila\textquotesingle{}}\NormalTok{,}
                            \DataTypeTok{host =} \StringTok{\textquotesingle{}localhost\textquotesingle{}}\NormalTok{)}

\CommentTok{\# List the tables available in the sakila database}
\KeywordTok{dbListTables}\NormalTok{(connection)}
\end{Highlighting}
\end{Shaded}

\begin{verbatim}
## [1] "actor"      "actor_info" "address"    "category"   "city"      
## [6] "country"
\end{verbatim}

\hypertarget{sending-a-query}{%
\subsection{Sending a query}\label{sending-a-query}}

We can query a table using the following code:

\begin{Shaded}
\begin{Highlighting}[]
\CommentTok{\# Query the "actor" table to get all the rows}
\NormalTok{result \textless{}{-}}\StringTok{ }\KeywordTok{dbSendQuery}\NormalTok{(}\DataTypeTok{conn=}\NormalTok{connection, }\DataTypeTok{statement=}\StringTok{"SELECT * FROM actor"}\NormalTok{) }\CommentTok{\# }\AlertTok{NOTE}\CommentTok{: SQL syntax is case{-}insensitive}

\CommentTok{\# Store the first 15 rows of the result in an R data frame object}
\NormalTok{actors \textless{}{-}}\StringTok{ }\KeywordTok{dbFetch}\NormalTok{(result, }\DataTypeTok{n=}\DecValTok{15}\NormalTok{)}

\CommentTok{\# Remove the unused results}
\KeywordTok{dbClearResult}\NormalTok{(result)}
\end{Highlighting}
\end{Shaded}

\begin{verbatim}
## [1] TRUE
\end{verbatim}

Note that the last line of code (\texttt{dbClearResult(result)}) clears
the fecched data and closes the query. It is now preferred to query a
table with the following code, which doesn't leave the query open:

\begin{Shaded}
\begin{Highlighting}[]
\CommentTok{\# Note {-} you can use single command below to send query and send result as data.frame in one command {-} doesn\textquotesingle{}t leave query open}
\NormalTok{actors \textless{}{-}}\StringTok{ }\KeywordTok{dbGetQuery}\NormalTok{(}\DataTypeTok{conn=}\NormalTok{connection, }\DataTypeTok{statement=}\StringTok{"SELECT * FROM actor"}\NormalTok{)}
\end{Highlighting}
\end{Shaded}

Let's take a quick look at the actors table:

\begin{Shaded}
\begin{Highlighting}[]
\KeywordTok{head}\NormalTok{(actors)}
\end{Highlighting}
\end{Shaded}

\begin{verbatim}
##   actor_id first_name    last_name         last_update
## 1        1   PENELOPE      GUINESS 2006-02-15 04:34:33
## 2        2       NICK     WAHLBERG 2006-02-15 04:34:33
## 3        3         ED        CHASE 2006-02-15 04:34:33
## 4        4   JENNIFER        DAVIS 2006-02-15 04:34:33
## 5        5     JOHNNY LOLLOBRIGIDA 2006-02-15 04:34:33
## 6        6      BETTE    NICHOLSON 2006-02-15 04:34:33
\end{verbatim}

Now, let's try getting the information for actors whose last name is
``TEMPLE'':

\begin{Shaded}
\begin{Highlighting}[]
\CommentTok{\# Note {-} you can use single command below to send query and send result as data.frame in one command {-} doesn\textquotesingle{}t leave query open}
\NormalTok{actors \textless{}{-}}\StringTok{ }\KeywordTok{dbGetQuery}\NormalTok{(}\DataTypeTok{conn=}\NormalTok{connection, }\DataTypeTok{statement=}\StringTok{"SELECT * FROM actor WHERE last\_name = \textquotesingle{}TEMPLE\textquotesingle{}"}\NormalTok{)}

\CommentTok{\# Print the table}
\NormalTok{(actors)}
\end{Highlighting}
\end{Shaded}

\begin{verbatim}
##   actor_id first_name last_name         last_update
## 1       53       MENA    TEMPLE 2006-02-15 04:34:33
## 2      149    RUSSELL    TEMPLE 2006-02-15 04:34:33
## 3      193       BURT    TEMPLE 2006-02-15 04:34:33
## 4      200      THORA    TEMPLE 2006-02-15 04:34:33
\end{verbatim}

\hypertarget{creating-a-table}{%
\subsection{Creating a table}\label{creating-a-table}}

In order to create a \texttt{MySQL} table, we first need to set our
permissions so that we can input data locally (see
\href{https://stackoverflow.com/questions/44288358/is-there-a-faster-way-to-upload-data-from-r-to-mysql}{this}
question for more information):

\begin{Shaded}
\begin{Highlighting}[]
\KeywordTok{dbSendQuery}\NormalTok{(}\DataTypeTok{conn=}\NormalTok{connection, }\DataTypeTok{statement=}\StringTok{"SET GLOBAL local\_infile = 1"}\NormalTok{)}
\end{Highlighting}
\end{Shaded}

\begin{verbatim}
## <MySQLResult:324305608,0,4>
\end{verbatim}

With that done, we are going to store the \texttt{mtcars} table on our
local \texttt{MySQL} server using the following code:

\begin{Shaded}
\begin{Highlighting}[]
\CommentTok{\# Upload mtcars table onto MySQL server}
\KeywordTok{dbWriteTable}\NormalTok{(connection, }\StringTok{"mtcars"}\NormalTok{, mtcars[, ], }\DataTypeTok{overwrite=}\OtherTok{TRUE}\NormalTok{)}
\end{Highlighting}
\end{Shaded}

\begin{verbatim}
## [1] TRUE
\end{verbatim}

\begin{Shaded}
\begin{Highlighting}[]
\CommentTok{\# Check the table is there}
\NormalTok{mtcars\_mysql \textless{}{-}}\StringTok{ }\KeywordTok{dbGetQuery}\NormalTok{(}\DataTypeTok{conn=}\NormalTok{connection, }\DataTypeTok{statement=}\StringTok{"SELECT * FROM mtcars"}\NormalTok{)}
\KeywordTok{head}\NormalTok{(mtcars\_mysql)}
\end{Highlighting}
\end{Shaded}

\begin{verbatim}
##           row_names  mpg cyl disp  hp drat    wt  qsec vs am gear carb
## 1         Mazda RX4 21.0   6  160 110 3.90 2.620 16.46  0  1    4    4
## 2     Mazda RX4 Wag 21.0   6  160 110 3.90 2.875 17.02  0  1    4    4
## 3        Datsun 710 22.8   4  108  93 3.85 2.320 18.61  1  1    4    1
## 4    Hornet 4 Drive 21.4   6  258 110 3.08 3.215 19.44  1  0    3    1
## 5 Hornet Sportabout 18.7   8  360 175 3.15 3.440 17.02  0  0    3    2
## 6           Valiant 18.1   6  225 105 2.76 3.460 20.22  1  0    3    1
\end{verbatim}

\hypertarget{updating-a-table}{%
\subsection{Updating a table}\label{updating-a-table}}

We can update a table on our \texttt{MySQL} server by sending a query,
for example we can use the code below to edit the data for cars with a
certain horse power:

\begin{Shaded}
\begin{Highlighting}[]
\CommentTok{\# Update data in the actor info table}
\KeywordTok{dbSendQuery}\NormalTok{(}\DataTypeTok{conn=}\NormalTok{connection, }\DataTypeTok{statement=}\StringTok{"UPDATE mtcars SET disp = 168.5 WHERE hp = 110"}\NormalTok{)}
\end{Highlighting}
\end{Shaded}

\begin{verbatim}
## <MySQLResult:11,0,9>
\end{verbatim}

\begin{Shaded}
\begin{Highlighting}[]
\CommentTok{\# Check the updated table}
\NormalTok{mtcars\_mysql \textless{}{-}}\StringTok{ }\KeywordTok{dbGetQuery}\NormalTok{(}\DataTypeTok{conn=}\NormalTok{connection, }\DataTypeTok{statement=}\StringTok{"SELECT * FROM mtcars"}\NormalTok{)}

\CommentTok{\# Take a look to see if the data has changed}
\KeywordTok{head}\NormalTok{(mtcars\_mysql[mtcars\_mysql}\OperatorTok{$}\NormalTok{hp }\OperatorTok{==}\StringTok{ }\DecValTok{110}\NormalTok{, ])}
\end{Highlighting}
\end{Shaded}

\begin{verbatim}
##        row_names  mpg cyl  disp  hp drat    wt  qsec vs am gear carb
## 1      Mazda RX4 21.0   6 168.5 110 3.90 2.620 16.46  0  1    4    4
## 2  Mazda RX4 Wag 21.0   6 168.5 110 3.90 2.875 17.02  0  1    4    4
## 4 Hornet 4 Drive 21.4   6 168.5 110 3.08 3.215 19.44  1  0    3    1
\end{verbatim}

\hypertarget{inserting-a-row-into-a-table}{%
\subsection{Inserting a row into a
table}\label{inserting-a-row-into-a-table}}

Using the code below, we can insert data for a new car into the
\texttt{mtcars} table on our \texttt{MySQL} server:

\begin{Shaded}
\begin{Highlighting}[]
\CommentTok{\# Insert a new row into table}
\KeywordTok{dbSendQuery}\NormalTok{(}\DataTypeTok{conn=}\NormalTok{connection,}
            \DataTypeTok{statement=}\StringTok{"INSERT INTO mtcars(row\_names, mpg, cyl, disp, hp, drat, wt, qsec, vs, am, gear, carb) values(\textquotesingle{}New Mazda RX4 Wag\textquotesingle{}, 21, 6, 168.5, 110, 3.9, 2.875, 17.02, 0, 1, 4, 4)"}
\NormalTok{            )}
\end{Highlighting}
\end{Shaded}

\begin{verbatim}
## <MySQLResult:1,0,11>
\end{verbatim}

\begin{Shaded}
\begin{Highlighting}[]
\CommentTok{\# Check the updated table}
\NormalTok{mtcars \textless{}{-}}\StringTok{ }\KeywordTok{dbGetQuery}\NormalTok{(}\DataTypeTok{conn=}\NormalTok{connection, }\DataTypeTok{statement=}\StringTok{"SELECT * FROM mtcars"}\NormalTok{)}
\KeywordTok{tail}\NormalTok{(mtcars)}
\end{Highlighting}
\end{Shaded}

\begin{verbatim}
##            row_names  mpg cyl  disp  hp drat    wt  qsec vs am gear carb
## 28      Lotus Europa 30.4   4  95.1 113 3.77 1.513 16.90  1  1    5    2
## 29    Ford Pantera L 15.8   8 351.0 264 4.22 3.170 14.50  0  1    5    4
## 30      Ferrari Dino 19.7   6 145.0 175 3.62 2.770 15.50  0  1    5    6
## 31     Maserati Bora 15.0   8 301.0 335 3.54 3.570 14.60  0  1    5    8
## 32        Volvo 142E 21.4   4 121.0 109 4.11 2.780 18.60  1  1    4    2
## 33 New Mazda RX4 Wag 21.0   6 168.5 110 3.90 2.875 17.02  0  1    4    4
\end{verbatim}

\hypertarget{removing-a-table}{%
\subsection{Removing a table}\label{removing-a-table}}

We can remove the \texttt{mtcars} table from our \texttt{MySQL} server
with the following code:

\begin{Shaded}
\begin{Highlighting}[]
\KeywordTok{dbSendQuery}\NormalTok{(connection, }\StringTok{\textquotesingle{}drop table if exists mtcars\textquotesingle{}}\NormalTok{)}
\end{Highlighting}
\end{Shaded}

\begin{verbatim}
## <MySQLResult:4,0,13>
\end{verbatim}

\hypertarget{closing-a-connection-to-the-mysql-server}{%
\subsection{\texorpdfstring{Closing a connection to the \texttt{MySQL}
server}{Closing a connection to the MySQL server}}\label{closing-a-connection-to-the-mysql-server}}

To wrap up, we can close our connection to the \texttt{MySQL} server
using the following code:

\begin{Shaded}
\begin{Highlighting}[]
\KeywordTok{dbDisconnect}\NormalTok{(}\DataTypeTok{conn=}\NormalTok{connection)}
\end{Highlighting}
\end{Shaded}

\begin{verbatim}
## [1] TRUE
\end{verbatim}

\hypertarget{the-end--}{%
\section{The end :-)}\label{the-end--}}

So, that's me done. This tutorial has hopefully got you started with
using the \texttt{RMySQL} R package. If you haven't already I would
recommend going and doing some more tutorials about \texttt{MySQL}.
\href{https://www.tutorialspoint.com/mysql/index.htm}{Here} is a good
one to start with.

As you'll have noticed, this tutorial has barely scratched the surface
of using \texttt{MySQL}. It is an extremely powerful language and one
that will be found in many Reproducible Analytical Pipelines and well
worth learning about.

\end{document}
